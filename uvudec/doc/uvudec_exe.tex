\section{uvudec Executable}
Describes theory behind the uvudec executable, the core of this project.

\subsection{Command line summary}
\label{sec::uvudec::command_line}

% TODO: get this from command output

\input{uvudec_exe_help}

% \input "|ls/b"


\begin{comment}

Usage: ./uvudec <args>
Args:
--verbose: verbose output.  Equivilent to --verbose=3
--verbose=<level>: set verbose level.  0 (none) - 3 (most)
--verbose-init: for selectivly debugging configuration file reading
--verbose-analysis: for selectivly debugging code analysis
--verbose-processing: for selectivly debugging code post-analysis
--verbose-printing: for selectivly debugging print routine
--config-language=<language>: default config interpreter language (plugins may require specific)
        python: use Python
        javascript: use javascript
--addr-min=<min>: minimum analysis address
--addr-max=<max>: maximum analysis address
--addr-exclude-min=<min>: minimum exclusion address
--addr-exclude-max=<max>: maximum exclusion address
--addr-comment: put comments on addresses
--addr-label: label addresses for jumping
--analysis-only[=<bool>]: only do analysis, don't print data
--analysis-address=<address>: only output analysis data for specified address
--flow-analysis=<type>: how to trace jump, calls
        linear: start at beginning, read all instructions linearly, then find jump/calls (default)
        trace: start at all vectors, analyze all segments called/branched recursivly
--opcode-usage: opcode usage count table
--analysis-dir=<dir>: create skeleton data suitible for stored analysis
--input=<file name>: source for data
--output=<file name>: output program (default: stdout)
--debug=<file name>: debug output (default: stdout)
--print-jumped-addresses=<bool>: whether to print information about jumped to addresses (*1)
--print-called-addresses=<bool>: whether to print information about called to addresses (*1)
--useless-ascii-art: append nifty ascii art headers to output files
--help: print this message and exit
--version: print version and exit

Special files: -: stdin
<bool>:
        true includes case insensitive "true", non-zero numbers (ie 1)
        false includes case insensitve "false", 0

*1: WARNING: currently slow, may be fixed in future releases

\end{comment}

\subsection{Basic usage}
\label{sec::uvudec::basic_usage}

Example:
./uvudec embedded.bin
Will print the disassembly/decompile from the image.  
Will try to automatically guess architecture based off of heuristics.
Currently the heuristic algorithm is blazing fast at O(0) and will assume you are using 8051.
This is trivial to fix, but I've been too lazy.

\subsection{Advanced}
\label{sec::uvudec::advanced}
